\chapter{Measurements (Voorkeur naar eigen informatieve titel)}

Hoofdstuk 4 Resultaten: in dit hoofdstuk wordt de presentatie van de data gegeven in de vorm van grafieken en/of korte functionele tabellen. Leg uit wat je gemeten hebt, waar de data staat (in dit hoofdstuk / paragraaf of bijlage). Leg de grafieken uit en geef vooral aan wat de lezer uit de grafiek kan afleiden. Maak ook een voorbeeldberekening (inclusief de gebruikte formules) n.a.v. de data met direct daarna de onzekerheidsberekening. Bij een eenvoudige berekening (vb. wet van Ohm) hoeft geen voorbeeld berekening opgenomen te worden. De rest kan in een tabel of in een extra kolom bij de datatabel. Zijn er voor de grafieken al berekeningen nodig dan komen de berekeningen direct na de data presentatie. Bespreek de trendlijnen, pas regressieanalyse toe en breng de koppeling met de theorie aan. Schijf eerst tekst voordat je een tabel of grafiek presenteert.

Bij meetresultaten alleen relevante tabellen en/of grafieken (zie ook bijlagen) zodat de leesbaarheid niet verstoord wordt. Bij voorkeur een grafiek in plaats van een tabel. Een grafiek laat in één oogopslag het verloop van de relatie tussen de grootheden zien. De tabel moet wel in het verslag opgenomen worden; dit kan in de tekst (kleine tabel) of in een bijlage.

\section{Uncertainty Calculations}

\section{Data Analytics}