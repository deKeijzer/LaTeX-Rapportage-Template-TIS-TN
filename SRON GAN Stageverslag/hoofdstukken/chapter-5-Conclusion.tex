\chapter{Conclusion and Discussion}
It is seen that the inpainting from the typical physical model is performed well. In general the inpainting of the spectra is done well, with little noise on the fit to the spectrum. This, combined with the contextual and perceptual loss being in balance, shows that the inpainting process is performed as expected.

The retrieved results in Table \ref{tab:single_simple_result}, when considering the errors from Table \ref{tab:simple_result}, are in agreement with the ground-truth values for H$_2$O, CO$_2$, CH$_4$, M$_P$, R$_P$, T$_P$, but not for CO. This is expected since CO is one of the hardest molecules to retrieve, considering it has little visible spectral signatures.


The results from the simple physical model in Table \ref{tab:simple_result} cannot be directly compared to the ExoGAN results from Table \ref{table:exogantestresults}, due to the different error analysis. However, CO and CH$_4$ having the largest error is in agreement with the results of ExoGAN. In general percentage errors are highly influenced by the scale of the values going into Equation \ref{eq:perc_error}. Therefor it is justified to prefer the absolute errors over the percentage errors. Taking into consideration the scales of the original parameters, e.g. the abundances which are sampled between 10$^{-8}$ and 10$^{-1}$. An absolute error of 0.6 - 0.7 dex could be considered a reasonable. This concludes that for the simple physical models, all features except CO and CH$_4$ are properly retrieved. 

When looking at the complex physical model, the first impression is exceptionally clear. There is a contrast between the retrieval results of the abundances and all other features. SRON-GAN is able to retrieve most abundances with reasonably errors, some of which on exceptionally low. Being able to retrieve abundances downwards to 10$^{-100}$ indicates that SRON-GAN has learned some underlying form of physics and chemistry. However, it does completely fail to retrieve any other feature. Notice that this conclusion is hard to make, due there being no literature to make comparisons too. However, the errors on the abundances should have a considerable positive effect on the run-time of a traditional retrieval, due to the constrains it gives on the features.

Most features which SRON-GAN fails to predict are understandable. They are hard to predict, even with traditional retrieval methods. However, SRON-GAN should be able to predict the planet radius directly out of the normalisations factors from the spectrum. The C/O ratio can be directly retrieved from the result of all the other abundances. Clearly, it fails to predict these features.


SRON-GAN not being able to predict these features indicates that the model has not been converged yet, after 8 days of training. During training $D(G(\mathbf{z}))$ seemed to get stuck at 0.75. This also indicates that the GAN has not converged yet. No noticeable changes to the loss functions have been made at this point of training, hence the model was expected to have stopped to learn. Which might indicate the need to increase the number of kernels or amount of deep layers within the network. However, making these changes did not lead to an improvement in results. This result is in agreement to \cite{mescheder2018training}, where the WGAN-GP loss function is found to not always fully converge. Fortunately there is still room for improvement. As seen in Figure \ref{fig:distribution}, lots of distributions are neither uniform or Gaussian distributed. This results in the neural networks having a hard time to properly learn these distributions, due to the information density being relatively high in a small area. In future work it is advised to transform these distributions non-linearly. Obviously this is not a guaranteed ticket to success. In this case the loss function did not converge. The discriminator might have simply ignored the features SRON-GAN failed to retrieve. Hence a different loss function might be an option. In general this setup seems to be unreliable, considering the difference between the results of \ref{fig:gan_results_small_gpu2} - \ref{fig:gan_results_small_gpu5}. On the positive side, notice how SRON-GAN is able to decently predict abundances up to $10^{-100}$-$10^{-60
}$ for C$_2$H$_2$ , C$_2$H$_4$, HCN, K, SO$_2$ and TiO. Abundances this small are hardly visible as signatures within the spectra. This indicates that SRON-GAN has, at least partly, learned the underlying physics and chemistry. However, further research has to be done to conclude this. This could e.g. be done by inpainting while masking the spectrum of all features except CH$_4$, CO$_2$, CO, H$_2$O, temperature profile, H$_2$, NH$_3$ and P. An additional overview of this work is given in Appendix C.

Summarising, at this stage SRON-GAN is able to retrieve the abundances in 16 seconds per spectrum. It fails to retrieve most other features. A promising result is that the retrieval performed by SRON-GAN on 10 real observations, mostly agree with the retrieval result performed by ARCiS. However, there is lots of room for improvement and future research. Be it by;
\begin{itemize}
\item Using a proper hyperparameter and architecture scan.
\item Non-linear transformation of the input data.
\item Using a different loss function.
\end{itemize}
GANs have shown potential, but it is clearly an active research area.