\chapter{Conclusion}
The research question as given in Chapter 1 is:
 \begin{itemize}
     \item  What is the charge carrier density of silver?
 \end{itemize}
The Hall coefficient of silver is determined to be:
    \begin{equation}
        R_H = (8.3 \pm 2) \cdot 10^{-11} \quad \text{[m$^3$C$^{-1}$]}
    \end{equation}
The literature value for the Hall coefficient of silver is \cite{apparatus_silver}:
    \begin{equation}
        \mid R_{H}_{ literature} \mid = (8,9 \pm 2) \cdot 10^{-11}  \quad \text{[m$^3$C$^{-1}$]}
    \end{equation}
The charge carrier density of silver is determined to be:
    \begin{equation}
        n = (7.4 \pm 2)\cdot 10^{28} \quad \text{[m$^{-3}$]}.
    \end{equation}
Of which the literature value is equal to \cite{apparatus_silver}:
    \begin{equation}
        n_{literature} = (6.6 \pm 2)\cdot 10^{28} \quad \text{[m$^{-3}$]}
    \end{equation}
It can be concluded that the measured Hall coefficient and charge carrier density of silver are in agreement with the literature values. The physical results for silver are also similar to the simulation results as presented in Chapter 4.5. Therefor it can be conclude that the used model is correct for silver. Moreover, there is no reason to believe that this model (Equation \ref{eq:Hall Voltage2}) is incorrect for other materials in different physical situations.